
\section{Introduction}

Let $\mathbf{F}\in\mathbb{K}\left[x\right]^{m\times n}$ be a matrix
of polynomials over a field $\mathbb{K}$ with $m\le n$. A (right)
nullspace basis $\mathbf{N}$ of $\mathbf{F}$ is said to be minimal
if it has the minimal possible column degrees among all right nullspace
bases. More generally, for a shift $\vec{s}=[s_{1},\dots,s_{n}]\in\mathbb{Z}^{n}$,
a nullspace basis $\mathbf{N}$ is said to be $\vec{s}$-minimal if
it has the minimal possible $\vec{s}$-column degrees among all nullspace
bases, or equivalently, the column degrees of 
\[
x^{\vec{s}}\mathbf{N}=\begin{bmatrix}s_{1}\\
 & \ddots\\
 &  & s_{n}
\end{bmatrix}\mathbf{N}
\]
 are the minimal possible among all nullspace bases of $\mathbf{F}$.

In this paper we present an algorithm for computing a minimal nullspace
basis with a cost of $O^{\sim}\left(n^{\omega}\left\lceil md/n\right\rceil \right)$
field operations in $\mathbb{K}$. The same algorithm can also compute
a $\vec{s}$-minimal nullspace basis of $\mathbf{F}$ with a cost
of $O^{\sim}(n^{\omega}\rho/m)$ if the entries of $\vec{s}$ bound
the corresponding column degrees of $\mathbf{F}$, where $\rho$ is
the sum of the $m$ largest entries of $\vec{s}$.

\begin{comment}
A key ingredient of this algorithm is another algorithm, also presented
in this paper, for computing a $\vec{s}$-minimal shifted nullspace
basis of an input matrix with row dimension $m$ and column dimension
$n$ satisfy $n\in\Theta(m)$, where the entries of the shift $\vec{s}$
bound the column degrees of $\mathbf{F}$ component-wise, that is,
each entry $s_{i}$ is greater than or equal to the degree of the
$i$-th column of $\mathbf{F}$. For example, the shift $\vec{s}$
can be simply the column degrees of $\mathbf{F}$. If $\xi$ is the
sum of the entries of $\vec{s}$, a $\vec{s}$-minimal nullspace basis
can be computed with a cost of $O^{\sim}(n^{\omega-1}\xi)$ field
operations.
\end{comment}


A key component of the algorithm is order basis (also known as minimal
approximant basis) computation, which are done using the algorithms
from \citet{Giorgi2003} and \citet{za2009}. We use order basis computation
to compute a partial nullspace basis, which also reduces the column
dimension of the problem. The problem can then be separated to two
subproblems of smaller row dimensions, which can then be handled in
the same way as the original problem. 

The remaining paper is structured as follows. Basic definitions and
properties of order bases and nullspace bases and order bases are
given in the next section. \prettyref{sec:Nullspace-Basis-Computation}
then describe the computation of nullspace bases in detail and analyze
its cost. This is followed by a conclusion along with a description
for topics for future research. 
