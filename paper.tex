%% LyX 2.0.0 created this file.  For more info, see http://www.lyx.org/.
%% Do not edit unless you really know what you are doing.
\documentclass[english,amsthm,seceqn,secthm]{elsart}
\usepackage[T1]{fontenc}
\usepackage[latin9]{inputenc}
\usepackage{color}
\usepackage{verbatim}
\usepackage{prettyref}
\usepackage{float}
\usepackage{amsmath}
\usepackage{amssymb}
\usepackage[authoryear]{natbib}

\makeatletter

%%%%%%%%%%%%%%%%%%%%%%%%%%%%%% LyX specific LaTeX commands.
\floatstyle{ruled}
\newfloat{algorithm}{tbp}{loa}
\providecommand{\algorithmname}{Algorithm}
\floatname{algorithm}{\protect\algorithmname}

%%%%%%%%%%%%%%%%%%%%%%%%%%%%%% Textclass specific LaTeX commands.
 \let\AND\relax 
 \usepackage{algorithmic}
\floatstyle{ruled}
\newfloat{algorithm}{tbp}{loa}
\floatname{algorithm}{Algorithm}
\usepackage{algorithmic}
\newcommand{\forbody}[1]{ #1 \ENDFOR }
\newcommand{\ifbody}[1]{ #1  \ENDIF}
\newcommand{\whilebody}[1]{ #1  \ENDWHILE}
\renewcommand{\algorithmicprint}{\textbf{draw}}
\renewcommand{\algorithmicrequire}{\textbf{Input:}}
\renewcommand{\algorithmicensure}{\textbf{Output:}}


%%%%%%%%%%%%%%%%%%%%%%%%%%%%%% User specified LaTeX commands.
\usepackage{yjsco}\journal{JournalofSymbolicComputation}

\renewcommand{\algorithmicrequire}{\textbf{Input:}}\renewcommand{\algorithmicensure}{\textbf{Output:}}
%\renewcommand{\algorithmicensure}{\textbf{if}}\renewcommand{\algorithmicensure}{\textbf{Uses:}}

\def\diag{\mbox{diag}}\def\cdeg{\qopname\relax n{cdeg}}
\def\MM{\qopname\relax n{MM}}\def\M{\qopname\relax n{M}}\def\ord{\qopname\relax n{ord}}

\def\StorjohannTransform{\qopname\relax n{StorjohannTransform}}\def\TransformUnbalanced{\qopname\relax n{TransformUnbalanced}}\def\rowDimension{\qopname\relax n{rowDimension}}\def\columnDimension{\qopname\relax n{columnDimension}}\DeclareMathOperator{\re}{rem}\DeclareMathOperator{\coeff}{coeff}\DeclareMathOperator{\lcoeff}{lcoeff}\def\mab{\qopname\relax n{orderBasis}}\def\mmab{\qopname\relax n{FastBasis}}\def\umab{\qopname\relax n{UnbalancedFastBasis}}\newcommand{\bb}{\\}
\def\mnb{\qopname\relax n{minNullspaceBasis}}
\makeatother

\makeatother

\usepackage{babel}
\begin{document}
\begin{frontmatter}

\title{Computing Minimal Nullspace Bases}


\author{Wei Zhou and George Labahn}


\address{{Cheriton School of Computer Science}\\
 {University of Waterloo},\\
 {Waterloo, Ontario, Canada}}

\ead{\{w2zhou,glabahn\}@uwaterloo.ca }

\selectlanguage{english}%
\begin{abstract}
In this paper we present a deterministic algorithm for the computation
of a minimal nullspace basis of an $m\times n$ input matrix of polynomials
over a field $\mathbb{K}$ with $m\le n$. This algorithm computes
a minimal nullspace basis of a degree $d$ input matrix with a cost
of $O^{\sim}\left(n^{\omega}\left\lceil md/n\right\rceil \right)$
field operations in $\mathbb{K}$. The same algorithm also works in
the more general situation on computing a shifted minimal nullspace
basis, with a given degree shift $\vec{s}\in\mathbb{Z}^{n}$ whose
entries bound the corresponding column degrees of the input matrix.
If $\rho$ is the sum of the $m$ largest entries of $\vec{s}$, then
a $\vec{s}$-minimal right nullspace basis can be computed with a
cost of $O^{\sim}(n^{\omega}\rho/m)$ field operations. 
\end{abstract}
\end{frontmatter}

\section{Introduction}

Let $\mathbf{F}\in\mathbb{K}\left[x\right]^{m\times n}$ be a matrix
of polynomials over a field $\mathbb{K}$ with $m\le n$. A (right)
nullspace basis $\mathbf{N}$ of $\mathbf{F}$ is said to be minimal
if it has the minimal possible column degrees among all right nullspace
bases. More generally, for a shift $\vec{s}=[s_{1},\dots,s_{n}]\in\mathbb{Z}^{n}$,
a nullspace basis $\mathbf{N}$ is said to be $\vec{s}$-minimal if
it has the minimal possible $\vec{s}$-column degrees among all nullspace
bases, or equivalently, the column degrees of 
\[
x^{\vec{s}}\mathbf{N}=\begin{bmatrix}s_{1}\\
 & \ddots\\
 &  & s_{n}
\end{bmatrix}\mathbf{N}
\]
 are the minimal possible among all nullspace bases of $\mathbf{F}$.

In this paper we present an algorithm for computing a minimal nullspace
basis with a cost of $O^{\sim}\left(n^{\omega}d\right)$. A key ingredient
of this algorithm is another algorithm, also presented in this paper,
for computing a $\vec{s}$-minimal shifted nullspace basis of an input
matrix with row dimension $m$ and column dimension $n$ satisfy $n\in\Theta(m)$,
where the entries of the shift $\vec{s}$ bound the column degrees
of $\mathbf{F}$ component-wise, that is, each entry $s_{i}$ is greater
than or equal to the degree of the $i$-th column of $\mathbf{F}$.
For example, the shift $\vec{s}$ can be simply the column degrees
of $\mathbf{F}$. If $\xi$ is the sum of the entries of $\vec{s}$,
a $\vec{s}$-minimal nullspace basis can be computed with a cost of
$O^{\sim}(n^{\omega-1}\xi)$ field operations.

A key component of the algorithm is order basis (also known as minimal
approximant basis) computation, which are done using the algorithms
from \citet{Giorgi2003} and \citet{za2009}. We use order basis computation
to compute a partial nullspace basis, which also reduces the column
dimension of the problem. The problem can then be separated to two
subproblems of smaller row dimensions, which can then be handled in
the same way as the original problem. 

The remaining paper is structured as follows. Basic definitions and
properties of nullspace bases and order bases are given in the next
section. \prettyref{sec:transform} provides an extension to Storjohann's
transformation to allow higher degree basis elements to be computed.
Based on this new transformation, \prettyref{sec:Order-Basis-Computation}
establishes a link between two Storjohann transformed problems of
different degrees, from which an recursive method and then an iterative
algorithm are derived. The time complexity is analyzed in the next
section. After this, \prettyref{sec:Unbalanced-Shift} describes an
algorithm which handles problems with a type of unbalanced shift.
This is followed by a conclusion along with a description for topics
for future research. 



\section{Preliminaries}

\label{sec:Background}

The computational cost in this paper is analyzed by bounding the number
of arithmetic operations (additions, subtractions, multiplications,
and divisions) in the coefficient field $\mathbb{K}$ on an algebraic
random access machine. We assume the cost of multiplying two polynomial
matrices with dimension $n$ and degree $d$ is $O^{\sim}(n^{\omega}d)$
field operations, where the multiplication exponent $\omega$ is assumed
to satisfy $2<\omega\le3$. We refer to the book by \citet{vonzurgathen}
for more details and reference about the cost of polynomial multiplication
and matrix multiplication.

In the remaining of this section, we provide some of the background
needed in order to understand the basic concepts and tools needed
for nullspace basis computation. This includes basic definitions and
a look at the size of the input and the output for computing such
bases.%
\begin{comment}
, which provide lower bounds for the computational cost 
\end{comment}
{} We also provide a brief introduction to order basis, which is a key
ingredient in our computation. 


\subsection{Order Basis}

Let $\mathbb{K}$ be a field, $\mathbf{F}\in\mathbb{K}\left[\left[x\right]\right]^{m\times n}$
a matrix of power series and $\vec{\sigma}=\left[\sigma_{1},\dots,\sigma_{m}\right]$
a vector of non-negative integers. 
\begin{defn}
A vector of polynomials $\mathbf{p}\in\mathbb{K}\left[x\right]^{n\times1}$
has \emph{order} $\left(\mathbf{F},\vec{\sigma}\right)$ (or \emph{order}
$\vec{\sigma}$ with respect to $\mathbf{F}$) if $\mathbf{F}\cdot\mathbf{p}\equiv\mathbf{0}\mod x^{\vec{\sigma}}$,
that is, 
\[
\mathbf{F}\cdot\mathbf{p}=x^{\vec{\sigma}}\mathbf{r}=\begin{bmatrix}x^{\sigma_{1}}\\
 & \ddots\\
 &  & x^{\sigma_{m}}
\end{bmatrix}\mathbf{r}
\]
 for some $\mathbf{r}\in\mathbb{K}\left[\left[x\right]\right]^{m\times1}$.
If $\vec{\sigma}=\left[\sigma,\dots,\sigma\right]$ is uniform, then
we say that $\mathbf{p}$ has order $\left(\mathbf{F},\sigma\right).$
\begin{comment}
The vector of power series $\mathbf{r}$ is called the order $\left(\mathbf{F},\sigma\right)$-residual
of \textbf{$\mathbf{p}$}. 
\end{comment}
{} The set of all order $\left(\mathbf{F},\vec{\sigma}\right)$ vectors
is a $\mathbb{K}\left[x\right]$-module denoted by $\left\langle \left(\mathbf{F},\vec{\sigma}\right)\right\rangle $. 
\end{defn}
An order basis for $\mathbf{F}$ and $\vec{\sigma}$ is simply a basis
for the module $\left\langle \left(\mathbf{F},\vec{\sigma}\right)\right\rangle $.
In this paper we compute those order bases having a type of minimality
degree condition (also referred to as a reduced order basis in \citep{BL1997}).
While minimality is often given in terms of the degrees alone it is
sometimes important to consider this in terms of shifted degrees \citep{BLV:jsc06}.

The shifted column degree of a column polynomial vector $\mathbf{p}$
with shift $\vec{s}=\left[s_{1},\dots,s_{n}\right]\in\mathbb{Z}^{n}$
is given by 
\[
\deg_{\vec{s}}\mathbf{p}=\max_{1\le i\le n}[\deg p^{\left(i\right)}+s_{i}]=\deg(x^{\vec{s}}\cdot\mathbf{p}).
\]
 We call this the \emph{$\vec{s}$-column degree}, or simply the \emph{$\vec{s}$-degree}
of $\mathbf{p}$. A shifted column degree defined this way is equivalent
to the notion of \emph{defect} commonly used in the literature. Our
definition of $\vec{s}$-degree is also equivalent to the notion of
$\mathbf{H}$-degree from \citep{BL1997} for $\mathbf{H}=x^{\vec{s}}$.
As in the uniform shift case, we say a matrix is \emph{$\vec{s}$-column
reduced} or \emph{$\vec{s}$-reduced} if its $\vec{s}$-degrees cannot
be decreased by unimodular column operations. More precisely, if $\mathbf{P}$
is a $\vec{s}$-column reduced and $[d_{1},\dots,d_{n}]$ are the
$\vec{s}$-degrees of columns of $\mathbf{P}$ sorted in nondecreasing
order, then $[d_{1},\dots,d_{n}]$ is lexicographically minimal among
all matrices right equivalent to $\mathbf{P}$. Note that a matrix
$\mathbf{P}$ is $\vec{s}$-column reduced if and only if $x^{\vec{s}}\cdot\mathbf{P}$
is column reduced\citep{BLV:1999,BLV:jsc06}. 

An \emph{order basis} \citep{BeLa94,BL1997} $\mathbf{P}$ of $\mathbf{F}$
with order $\vec{\sigma}$ and shift $\vec{s}$, or simply an $\left(\mathbf{F},\vec{\sigma},\vec{s}\right)$-basis,
is a basis for the module $\left\langle \left(\mathbf{F},\vec{\sigma}\right)\right\rangle $
\begin{comment}
\[
\left\langle \left(\mathbf{F},\vec{\sigma}\right)\right\rangle =\{\mathbf{p}\in\mathbb{K}\left[x\right]^{n\times1}\|\mathbf{F}\cdot\mathbf{p}=x^{\vec{\sigma}}\mathbf{r},\mathbf{r}\in\mathbb{K}[[x]]^{m\times1}\}
\]
\end{comment}
{} having minimal $\vec{s}$-column degrees. If $\vec{\sigma}=\left[\sigma,\dots,\sigma\right]$
are constant vectors then we simply write $\left(\mathbf{F},\sigma,\vec{s}\right)$-basis.
The precise definition of an $\left(\mathbf{F},\vec{\sigma},\vec{s}\right)$-basis
is as follows.
\begin{defn}
A polynomial matrix $\mathbf{P}$ is an order basis of $\mathbf{F}$
of order $\sigma$ and shift $\vec{s}$, denoted by $\left(\mathbf{F},\vec{\sigma},\vec{s}\right)$-basis,
if the following properties hold:
\begin{enumerate}
\item $\mathbf{P}$ is a nonsingular matrix of dimension $n$.
\item $\mathbf{P}$ is $\vec{s}$-column reduced. 
\item $\mathbf{P}$ has order $\left(\mathbf{F},\vec{\sigma}\right)$ (or
equivalently, each column of $\mathbf{P}$ is in $\left\langle (\mathbf{F},\vec{\sigma})\right\rangle $). 
\item Any $\mathbf{q}\in\left\langle \left(\mathbf{F},\vec{\sigma}\right)\right\rangle $
can be expressed as a linear combination of the columns of $\mathbf{P}$,
given by $\mathbf{P}^{-1}\mathbf{q}$. 
\end{enumerate}
\end{defn}
The following is a special case of the main result from \citep{BL1997},
a useful result for computing order bases. It can be also easily extended
to compute nullspace basis. We have included a proof for completeness.
\begin{lem}
For an input matrix $\mathbf{F}\in\mathbb{K}\left[x\right]^{m\times n}$,
an order vector $\vec{\sigma}$, and a shift vector $\vec{s}$, if
$\mathbf{P}$ is a $\left(\mathbf{F},\vec{\sigma},\vec{s}\right)$-basis
with $\vec{s}$-column degrees $\vec{t}$, and $\mathbf{Q}$ is a
$(\mathbf{F}\mathbf{P},\vec{\tau},\vec{t})$ -basis with $\vec{t}$-column
degrees $\vec{u}$, where $\vec{\tau}\ge\vec{\sigma}$ component-wise,
then $\mathbf{P}\mathbf{Q}$ is a $(\mathbf{F},\vec{\tau},\vec{s})$-basis
with $\vec{s}$-column degrees $\vec{u}$.\end{lem}
\begin{pf}
It is clear that $\mathbf{P}\mathbf{Q}$ has order $(\mathbf{F},\vec{\tau})$.
We now show that $\mathbf{P}\mathbf{Q}$ is $\vec{s}$-column reduced
and has $\vec{s}$-column degrees $\vec{u}$, or equivalently, $x^{\vec{s}}\mathbf{P}\mathbf{Q}$
is column reduced and has column degrees $\vec{u}$. Notice that $x^{\vec{s}}\mathbf{P}$
has column degrees $\vec{t}$ and a full rank leading column coefficient
matrix $P$. Hence $x^{\vec{s}}\mathbf{P}x^{-\vec{t}}$ has column
degrees $\left[0,\dots0\right]$. Similarly, $x^{\vec{t}}\mathbf{Q}x^{-\vec{u}}$
has column degrees $[0,\dots,0]$ and a full rank leading column coefficient
matrix $Q$. Therefore, $x^{\vec{s}}\mathbf{P}x^{-\vec{t}}x^{\vec{t}}\mathbf{Q}x^{-\vec{u}}=x^{\vec{s}}\mathbf{P}\mathbf{Q}x^{-\vec{u}}$
has column degrees $[0,\dots,0]$ and a full rank leading column coefficient
matrix $PQ$, implying that $x^{\vec{s}}\mathbf{P}\mathbf{Q}$ is
column reduced and that $x^{\vec{s}}\mathbf{P}\mathbf{Q}$ has column
degrees $\vec{u}$, or equivalently, the $\vec{s}$-column degrees
of $\mathbf{PQ}$ is $\vec{u}$.

It remains to show that any $\mathbf{t}\in\left\langle \left(\mathbf{F},\vec{\tau}\right)\right\rangle $
is generated by the columns of $\mathbf{PQ}$. Since $\mathbf{t}\in\left\langle \left(\mathbf{F},\vec{\sigma}\right)\right\rangle $,
it is generated by the $\left(\mathbf{F},\vec{\sigma}\right)$-basis
$\mathbf{P}$, that is, $\mathbf{t}=\mathbf{P}\mathbf{a}$ for $\mathbf{a}=\mathbf{P}^{-1}\mathbf{t}\in\mathbb{K}\left[x\right]^{n}$.
Also, $\mathbf{t}\in\left\langle \left(\mathbf{F},\vec{\tau}\right)\right\rangle $
implies that $\mathbf{a}\in\left\langle \left(\mathbf{FP},\vec{\tau}\right)\right\rangle $
since $\mathbf{F}\mathbf{P}\mathbf{a}=\mathbf{F}\mathbf{t}\equiv0\mod x^{\vec{\tau}}$.
It follows that $\mathbf{a}=\mathbf{Q}\mathbf{b}$ for $\mathbf{b}=\mathbf{Q}^{-1}\mathbf{a}\in\mathbb{K}\left[x\right]^{n}$.
Therefore, $\mathbf{a}=\mathbf{P}^{-1}\mathbf{t}=\mathbf{Q}\mathbf{b}$,
which gives $\mathbf{t}=\mathbf{P}\mathbf{Q}\mathbf{b}$.
\end{pf}

\subsection{Nullspace Basis}
\begin{defn}
A polynomial matrix $\mathbf{N}$ is a $\vec{s}$-minimal nullspace
basis of $\mathbf{F}\in\mathbb{K}\left[x\right]^{m\times n}$ if the
following properties hold:\end{defn}
\begin{enumerate}
\item $\mathbf{N}$ is full-rank.
\item $\mathbf{N}$ is $\vec{s}$-column reduced. 
\item $\mathbf{N}$ satisfies $\mathbf{F}\mathbf{N}=0$.
\item Any $\mathbf{q}\in\mathbb{K}\left[x\right]^{n}$ satisfying $\mathbf{F}\mathbf{q}=0$
can be expressed as a linear combination of the columns of $\mathbf{N}$,
that is, there exists some polynomial vector $\mathbf{a}$ such that
$\mathbf{q}=\mathbf{N}\mathbf{a}$.
\end{enumerate}
\begin{comment}
Note that the module $\left\langle \left(\mathbf{F},\vec{\sigma}\right)\right\rangle $
does not depend on the shift $\vec{s}$. 
\end{comment}


Although we allow different orders for each row in this definition,
we focus on order basis computation problems having uniform order.
However special cases of non-uniform order problems are still needed
in our analysis. We also assume $m\le n$ for simplicity. The case
of $m>n$ can be transformed to the case of $m\le n$ by compression
\citep{storjohann-villard:2005}. We further assume, without any loss
of generality, that $n/m$ and $\sigma$ are powers of two. This can
be achieved by padding zero rows to the input matrix and multiplying
it by some power of $x$.

From \citep{BL1997} we have the following lemma. 
\begin{lem}
\label{lem:orderBasisProperty} 

\label{lem:orderBasisEquivalence}The following are equivalent for
a polynomial matrix \textbf{$\mathbf{P}$}: 
\begin{enumerate}
\item $\mathbf{P}$ is a $\left(\mathbf{F},\vec{\sigma},\vec{s}\right)$-basis. 
\item $\mathbf{P}$ is comprised of a set of $n$ minimal $\vec{s}$-degree
polynomial vectors that are linearly independent and each having order
$\left(\mathbf{F},\vec{\sigma}\right)$. 
\item \label{enu:reduced+generator}$\mathbf{P}$ does not contain a zero
column, has order $\left(\mathbf{F},\vec{\sigma}\right)$, is $\vec{s}$-column
reduced, and any $\mathbf{q}\in\left\langle \left(\mathbf{F},\vec{\sigma}\right)\right\rangle $
can be expressed as a linear combination of the columns of $\mathbf{P}$. 
\end{enumerate}
\end{lem}
In some cases an entire order basis is unnecessary and instead one
looks for a minimal basis that generates only the elements of $\left\langle \left(\mathbf{F},\vec{\sigma}\right)\right\rangle $
with $\vec{s}$-degrees bounded by a given $\delta$. Such a minimal
basis is a partial $\left(\mathbf{F},\vec{\sigma},\vec{s}\right)$-basis
comprised of elements of a $\left(\mathbf{F},\vec{\sigma},\vec{s}\right)$-basis
with $\vec{s}$-degrees bounded by $\delta$. This is called a \emph{minbasis}
in \citet{Storjohann:2006}. 
\begin{defn}
\label{def:genset} Let $\left\langle \left(\mathbf{F},\vec{\sigma},\vec{s}\right)\right\rangle _{\delta}\subset\left\langle \left(\mathbf{F},\vec{\sigma}\right)\right\rangle $
denote the set of order $\left(\mathbf{F},\vec{\sigma}\right)$ polynomial
vectors with $\vec{s}$-degree bounded by $\delta$. A $\left(\mathbf{F},\vec{\sigma},\vec{s}\right)_{\delta}$-basis
is a polynomial matrix $\mathbf{P}$ not containing a zero column
and satisfying: 
\begin{enumerate}
\item $\mathbf{P}$ has order $\left(\mathbf{F},\vec{\sigma}\right).$ 
\item Any element of $\left\langle \left(\mathbf{F},\vec{\sigma},\vec{s}\right)\right\rangle _{\delta}$
can be expressed as a linear combination of the columns of $\mathbf{P}$. 
\item $\mathbf{P}$ is $\vec{s}$-column reduced. 
\end{enumerate}
\end{defn}
\begin{comment}
As before, the linear combination here is in fact unique. 
\end{comment}
A $\left(\mathbf{F},\vec{\sigma},\vec{s}\right)_{\delta}$-basis is,
in general, not square unless $\delta$ is large enough to contain
all $n$ basis elements in which case it is a complete $\left(\mathbf{F},\vec{\sigma},\vec{s}\right)$-basis.


\subsection{Computing Minimal Nullspace Basis Via Order Basis}

Minimal nullspace bases can be directly computed via order basis computation
using the following fact: If the order $\sigma$ of a $\left(\mathbf{F},\sigma,\vec{s}\right)$-basis
$\mathbf{P}$ is high enough, then $\mathbf{P}$ contains a $\vec{s}$-minimal
nullspace basis $\mathbf{N}$. However, this approach usually requires
the order $\sigma$ to be quite high. For example, if $\mathbf{F}$
has degree $d$ and $\vec{s}$ is uniform, its minimal nullspace bases
can have degree up to $md$, hence the order $\sigma$ need to be
set to $d+md$ in the order basis computation to fully compute a minimal
nullspace basis. Computing such a $\left(\mathbf{F},d+md\right)$-basis
cost $O^{\sim}\left(n^{\omega-1}\left\lceil m^{2}d/n\right\rceil \right)$
using the algorithm from \citep{za2009}. We can see from this cost
that there is room for improvement when $m$ is large. For example,
in the worst case when $m\in\Theta\left(n\right)$, this cost would
be $O^{\sim}\left(n^{\omega+1}d\right)$. We can also locate the root
cause for this inefficiency in the actual order basis computation.
Notice that when $m$ is large, the nullspace basis, which can have
a column dimension of $n-m$, is a much smaller subset of the order
basis computed. Hence much effort is put in the computation of order
basis elements that are not part of a nullspace basis. The key to
reduce the cost is therefore to reduce such computation of unneeded
order basis elements.



\section{Computation of Minimal Nullspace Basis}

\label{sec:Nullspace-Basis-Computation}

In this section, we describe an algorithm for computing a shifted
minimal nullspace basis, and then analyze its cost. The algorithm
uses two computation processes recursively. The first process described
in \prettyref{sub:continueComputingNullspaceBasisByColumns} uses
an order basis computation to compute a subset of nullspace basis
elements of lower degree, and results in a new problem of lower column
dimension. The second process described in \prettyref{sub:continueComputingNullspaceBasisByRows}
reduces the row dimension of the problem by computing a nullspace
basis of a submatrix formed by a subset of the rows of the input matrix.
We require the shift $\vec{s}$ to bound the column degrees of $\mathbf{F}$
component-wise. For example, we can set each entry of $\vec{s}$ to
be the corresponding column degree of $\mathbf{F}$, or we can simply
set each entry of $\vec{s}$ to be the maximum column degree of $\mathbf{F}$.
As it may become evident later in this section, this condition on
the shift is very useful, as it provides a simple bound on the column
degrees of $\mathbf{F}\mathbf{A}$ for any polynomial matrix $\mathbf{A}$.
\begin{lem}
\label{lem:boundOnDegreesOfFA}Let $\vec{s}$ be a shift with entries
bound the corresponding column degrees of $\mathbf{F}$ component-wise.
Then for any polynomial matrix $\mathbf{A}$, the column degrees of
$\mathbf{FA}$ are bounded component-wise by the $\vec{s}$-column
degrees of $\mathbf{A}$.
\end{lem}
A key requirement of efficient computation is making sure that the
intermediate computations do not blow up. This requirement is satisfied
by the existence of a bound $\xi=\sum\vec{s}$, the sum of all entries
of the initial input shift $\vec{s}$, which also bounds the sum of
all entries of the input shift of all subproblems throughout the computation,
as can be seen later in \prettyref{lem:boundOfSumOfShiftedDegreesOfOrderBasis},
\prettyref{lem:boundOfSumOfShiftedDegreesOfNullspaceBasis}, and \prettyref{lem:boundOnShiftedDegrees}.
\begin{lem}
\label{lem:boundOfSumOfShiftedDegreesOfOrderBasis}The sum of $\vec{s}$-column
degrees of an $(\mathbf{F},\vec{s},\sigma)$-basis $\mathbf{P}$ is
at most $\xi+k\sigma$, where $k$ is the rank of $\mathbf{F}$.\end{lem}
\begin{pf}
The sum is $\xi$ at order 0 and increases by at most $k$ for each
order increase.\end{pf}
\begin{lem}
\label{lem:boundOfSumOfShiftedDegreesOfNullspaceBasis}Given an $\mathbf{F}\in\mathbb{K}\left[x\right]^{m\times n}$,
and a shift $\vec{s}\in\mathbb{Z}_{\ge0}$ with entries bounding the
corresponding column degrees of $\mathbf{F}$. The sum of the $\vec{s}$-column
degrees of any $\vec{s}$-minimal nullspace basis of $\mathbf{F}$
is bounded by $\xi=\sum\vec{s}$.\end{lem}
\begin{pf}
Let $\mathbf{P}$ be a $(\mathbf{F},\vec{s},\sigma)$-basis with high
enough order $\sigma$ so $\mathbf{P}=\left[\mathbf{N},\bar{\mathbf{N}}\right]$
contains a complete null space basis $\mathbf{N}$ of $\mathbf{F}$.
Let $k$ be the column dimension of $\bar{\mathbf{N}}$ or equivalently,
the rank of $\mathbf{F}$. By \prettyref{lem:boundOfSumOfShiftedDegreesOfOrderBasis}
the sum of the $\vec{s}$-column degrees of $\mathbf{P}$ is at most
$\xi+k\sigma$. Also by \prettyref{lem:boundOnDegreesOfFA} the sum
of the $\vec{s}$-column degrees of $\bar{\mathbf{N}}$ is greater
than or equal to the sum of the column degrees of $\mathbf{F}\bar{\mathbf{N}}$,
which is at least $k\sigma$, since every column of $\mathbf{F}\bar{\mathbf{N}}$
is nonzero and has order $\sigma$. So the sum of the $\vec{s}$-column
degrees of $\mathbf{N}$ is bounded by $\xi+k\sigma-k\sigma=\xi$.
\end{pf}
For simplicity, we also assume without loss of generality that the
columns of $\mathbf{F}$ and the corresponding entries of $\vec{s}=\left[s_{1},\dots,s_{n}\right]$
are arranged so that the entries of $\vec{s}$ in increasing orders.
Let $\xi=\sum\vec{s}$ be the sum of the entries of $\vec{s}$. It
provides a convenient bound throughout the computation. Let $\rho=\sum_{n-m+1}^{n}s_{i}$
be the sum of $m$ largest entries of $\vec{s}$, and $s=\rho/m$
be their average. The algorithm we present in this section computes
a $\vec{s}$-minimal null space basis $\mathbf{N}$ with a cost of
$g(n,\rho)=O^{\sim}(n^{\omega}s)$. For uniform shift $\vec{s}=\left[s,\dots,s\right]$,
we improve this later to $O^{\sim}\left(n^{\omega}\left\lceil ms/n\right\rceil \right)=O^{\sim}\left(n^{\omega}\left\lceil \rho/n\right\rceil \right)=O^{\sim}\left(n^{\omega-1}\left\lceil m\xi/n\right\rceil \right)$.
\begin{comment}
For general unbalanced shifts, the cost improvement requires the ability
to efficiently compute order basis with general shifts.
\end{comment}



\subsection{\label{sub:continueComputingNullspaceBasisByColumns}Reducing the
column dimension via order basis computation}

First, we compute an $\left(\mathbf{F},3s,\vec{s}\right)$-basis $\mathbf{P}$,
which can be done with a cost $O^{\sim}\left(n^{\omega}s\right)$
using the algorithm from \citet{Giorgi2003}. Note that if $\vec{s}$
is balanced, then we can compute with a cost $O^{\sim}\left(n^{\omega}\left\lceil \rho/n\right\rceil \right)$
using the algorithm from \textbf{\citet{za2009}}. We claim at most
$3m/2$ columns of \textbf{$\mathbf{P}$} are not nullspace basis
elements of $\mathbf{F}$.
\begin{lem}
\label{lem:dimensionOfPartialNullspaceBasisBasedOnOrder} Let $\mathbf{P}=[\mathbf{P}_{1},\mathbf{P}_{2}]$
be a $(\mathbf{F},3s,\vec{s})$-basis with $\mathbf{P}_{1}$ containing
all columns $\mathbf{n}$ of $\mathbf{P}$ satisfying $\mathbf{F}\mathbf{n}=0$.
Then the column dimension $\kappa$ of $\mathbf{P}_{2}$ is less than
$3m/2.$\end{lem}
\begin{pf}
Any column $\mathbf{p}$ of $\mathbf{P}_{2}$ must have $\vec{s}$-column
degree at least $\sigma$, so their sum must satisfy $\sum\deg_{\vec{s}}\mathbf{P}_{2}\ge\kappa\sigma$.
Recall from \prettyref{lem:boundOfSumOfShiftedDegreesOfOrderBasis}
that the sum of $\vec{s}$-column degrees of the columns of $\mathbf{P}$
satisfies $\sum\deg_{\vec{s}}\mathbf{P}\le\sum\vec{s}+m\sigma$, so
the sum of $\vec{s}$-column degrees of the columns of $\mathbf{P}_{1}$
must satisfy $\sum\deg_{\vec{s}}\mathbf{P}_{1}\le\sum\vec{s}+m\sigma-\kappa\sigma$.
On the other hand, the lowest possible value of $\sum\deg_{\vec{s}}\mathbf{P}_{1}$
is $\sum_{i=1}^{n-\kappa}s_{i}$ (which occurs when $\mathbf{P}_{1}=\left[\mathbf{I},0\right]^{T}$)
based on the assumption that the entries of $\vec{s}=\left[s_{1},\dots,s_{n}\right]$
are arranged in increasing order. It follows that $\sum\vec{s}+m\sigma-\kappa\sigma\ge\sum_{i=1}^{n-\kappa}s_{i}$,
or $m\sigma\ge\kappa\sigma-\left(\sum\vec{s}-\sum_{i=1}^{n-\kappa}s_{i}\right)$
after rearrangement. Combining this with the fact that $\sum\vec{s}-\sum_{i=1}^{n-\kappa}s_{i}\le\kappa s$,
we get $m\sigma\ge\kappa\sigma-\kappa s$, which gives $\kappa\le m\sigma/(\sigma-s)$
for $\sigma>s$. If we set $\sigma=3s$, we get $\kappa\le3m/2$.
\end{pf}
Let $\left[\mathbf{P}_{1},\mathbf{P}_{2}\right]=\mathbf{P}$ with
$\mathbf{P}_{1}$ consists of the nullspace basis elements computed.
Then the residual $\mathbf{F}\mathbf{P}=\left[\mathbf{0},\mathbf{F}\mathbf{P}_{2}\right]$
can be used to compute the remaining nullspace basis elements. Before
showing the remaining nullspace basis elements can be computed this
way, we first need to ensure that the matrix multiplication $\mathbf{F}\mathbf{P}_{2}$
can be done efficiently. This is a consequence of the following more
general result on multiplying matrices with nonuniform degrees. It
is later used for multiplying other matrices as well.
\begin{lem}
\label{lem:multiplyUnbalancedMatrices}Given $\mathbf{A}\in\mathbb{K}\left[x\right]^{m\times n}$,
and a shift $\vec{s}$ with entries bound the column degrees of $\mathbf{A}$.
Let $\xi$ be the sum of the entries of $\vec{s}$. Let $\mathbf{B}\in\mathbb{K}\left[x\right]^{n\times k}$
with $k\in O\left(m\right)$ and the sum $\theta$ of its $\vec{s}$-column
degrees satisfies $\theta\in O\left(\xi\right)$. Then we can multiply
$\mathbf{A}$ and $\mathbf{B}$ with a cost of $O^{\sim}(nm^{\omega-2}\xi)$. \end{lem}
\begin{pf}
First, we divide the matrix $\mathbf{B}$ to column blocks according
to the $\vec{s}$-column degrees of its columns. Let 
\[
\left[\begin{array}{cccc}
\mathbf{B}^{\left(\log m\right)} & \mathbf{B}^{\left(\log m-1\right)} & \cdots & \mathbf{B}^{\left(1\right)}\end{array}\right]=\mathbf{B},
\]
with $\mathbf{B}^{\left(\log m\right)}$, $\mathbf{B}^{\left(\log m-1\right)},$
$\mathbf{B}^{\left(\log m-2\right)}$, ... , $\mathbf{B}^{\left(1\right)}$
having $\vec{s}$-column degrees in the range $\left[0,2\xi/m\right]$,
$(2\xi/m,4\xi/m]$, $(4\xi/m,8\xi/m]$, ..., $(\xi/2,\theta]$ respectively.
We multiply $\mathbf{A}$ with each $\mathbf{B}^{\left(i\right)}$
separately. 

We also divide the matrix $\mathbf{A}$ to column blocks and each
matrix $\mathbf{B}^{\left(i\right)}$ to row blocks according to the
size of the corresponding entries in $\vec{s}$. Let 
\begin{align*}
\left[\begin{array}{cccc}
\vec{s}_{\log m} & \vec{s}_{\log m-1} & \cdots & \vec{s}_{1}\end{array}\right] & =\vec{s}\\
\left[\begin{array}{cccc}
\mathbf{A}_{\log m} & \mathbf{A}_{\log m-1} & \cdots & \mathbf{A}_{1}\end{array}\right] & =\mathbf{A}\\
\left[\begin{array}{cccc}
\mathbf{B}_{\log m}^{\left(\log m\right)} & \mathbf{B}_{\log m}^{\left(\log m-1\right)} & \cdots & \mathbf{B}_{\log m}^{\left(1\right)}\\
\vdots &  &  & \vdots\\
\mathbf{B}_{1}^{\left(\log m\right)} & \mathbf{B}_{1}^{\left(\log m-1\right)} & \cdots & \mathbf{B}_{1}^{\left(1\right)}
\end{array}\right] & =\left[\begin{array}{cccc}
\mathbf{B}^{\left(\log m\right)} & \mathbf{B}^{\left(\log m-1\right)} & \cdots & \mathbf{B}^{\left(1\right)}\end{array}\right]=\mathbf{B}
\end{align*}
 with $\vec{s}_{\log m},\vec{s}_{\log m-1},\dots,\vec{s}_{\log m1}$
having entries in the range $\left[0,2\xi/m\right]$, $(2\xi/m,4\xi/m]$,
$(4\xi/m,8\xi/m]$, ..., $(\xi/2,\xi]$ respectively, the column dimension
of $\mathbf{A}_{j}$ and the row dimension of $\mathbf{B}_{j}^{\left(i\right)}$
match that of $\vec{s}_{j}$ for $j$ from $1$ to $\log m$.

First consider the multiplication 
\[
\mathbf{A}\mathbf{B}^{\left(1\right)}=\left[\begin{array}{cccc}
\mathbf{A}_{\log m} & \mathbf{A}_{\log m-1} & \cdots & \mathbf{A}_{1}\end{array}\right]\left[\begin{array}{l}
\mathbf{B}_{\log m}^{\left(1\right)}\\
\mathbf{B}_{\log m-1}^{\left(1\right)}\\
\vdots\\
\mathbf{B}_{1}^{\left(1\right)}
\end{array}\right].
\]
Note that there are $O\left(1\right)$ columns in $\mathbf{B}^{(1)}$
since $\theta\in O\left(\xi\right)$. We do this in $\log m$ steps.
At step $j$ for $j$ from $1$ to $\log m$ we multiply $\mathbf{A}_{j}$
and $\mathbf{B}_{j}^{(1)}$. The column dimension of $\mathbf{A}_{j}$,
which is the same as the row dimension of $\mathbf{B}_{j}^{(1)}$,
is less than $2^{j}$. The degree of $\mathbf{B}_{j}^{(1)}$ is $O\left(\xi\right)$.
To use fast multiplication, we expand $\mathbf{B}_{j}^{(1)}$ to a
matrix $\bar{\mathbf{B}}_{j}^{(1)}$ with degree less than $\delta\in\Theta(\xi/2^{j})$
and column dimension $q\in O(2^{j})$ as follows. Write 
\[
\mathbf{B}_{j}^{(1)}=\mathbf{B}_{j,0}^{(1)}+\mathbf{B}_{j,1}^{(1)}x^{\delta}+\dots+\mathbf{B}_{j,q-1}^{(1)}x^{\delta(q-1)}=\sum_{k=0}^{q-1}\mathbf{B}_{j,k}^{(1)}x^{\delta k}
\]
 with each $\mathbf{B}_{j,k}^{(1)}$ having degree less than $\delta.$
Set $\bar{\mathbf{B}}_{j}^{(1)}=\left[\mathbf{B}_{j,0}^{(1)},\mathbf{B}_{j,1}^{(1)},\dots,\mathbf{B}_{j,q-1}^{(1)}\right]$.
We can then multiply $\mathbf{A}_{j}$, which has dimension $m\times O(2^{j})$
for $j<\log m$, and $\bar{\mathbf{B}}_{j}^{(1)}$, which has dimension
$O(2^{j})\times O(2^{j})$ for $j<\log m$, with a cost of $O^{\sim}\left((m/2^{j})\left(2^{j}\right)^{\omega}\xi/2^{j}\right)=O^{\sim}\left(\left(2^{j}\right)^{\omega-2}m\xi\right)\subset O^{\sim}\left(m^{\omega-1}\xi\right)\subset O^{\sim}(nm^{\omega-2}\xi)$.
For $j=\log m$, $\mathbf{A}_{j}$ has dimension $m\times O\left(n\right)$,
$\bar{\mathbf{B}}_{j}^{\left(1\right)}$ has dimension $O\left(n\right)\times m$,
and their degrees are $O\left(\xi/m\right)$. Hence they can be multiplied
with a cost of $O^{\sim}\left((n/m)m^{\omega}(\xi/m)\right)=O^{\sim}\left(nm^{\omega-2}\xi\right)$.
Adding up the columns of $\mathbf{A}_{j}\bar{\mathbf{B}}_{j}^{(1)}$
gives $\mathbf{A}_{j}\mathbf{B}_{j}^{(1)}$ and cost $O(m\xi)$. Therefore,
multiplying $\mathbf{A}$ and $\mathbf{B}^{(1)}$ over $\log(n)$
steps cost $O^{\sim}\left(nm^{\omega-2}\xi\right)$. 

Next we multiply $\mathbf{A}$ with $\mathbf{B}^{(2)}$. We proceed
in the same way as before, but notice that the columns of $\mathbf{A}$
whose degrees are higher than $\xi/2$ are no longer needed since
the corresponding rows in $\mathbf{B}^{(2)}$ must be 0, as the $\vec{s}$-column
degree of $\mathbf{B}^{(2)}$ is bounded by $\xi/2$. Multiplying
$\mathbf{A}$ and $\mathbf{B}^{(2)}$ also cost $O^{\sim}\left(nm^{\omega-2}\xi\right)$.

Continue doing this, it cost $O^{\sim}\left(nm^{\omega-2}\xi\right)$.
to multiply $\mathbf{A}$ with the columns $\mathbf{B}^{(i)}$ for
$i$ from $1$ to $\log m$.  As before, notice that the columns of
$\mathbf{A}$ whose degrees are higher than $\xi/2^{i-1}$ are not
needed, as $\vec{s}$-column degrees of $\mathbf{B}^{(i)}$ is bounded
by $\xi/2^{i-1}$. The overall cost for $i$ from 1 to $\log m$ is
$O^{\sim}\left(nm^{\omega-2}\xi\right)$ to multiply $\mathbf{A}$
and $\mathbf{B}$. 
\end{pf}
With \prettyref{lem:multiplyUnbalancedMatrices}, we can now do the
multiplication $\mathbf{F}\mathbf{P}_{2}$ efficiently.
\begin{cor}
\label{cor:multiplyingFP2}The multiplication of $\mathbf{F}$ and
$\mathbf{P}_{2}$ can be done with a cost of $O^{\sim}\left(nm^{\omega-2}\xi\right)$.\end{cor}
\begin{pf}
Since $\mathbf{P}=[\mathbf{P}_{1},\mathbf{P}_{2}]$ is a $(\mathbf{F},3s,\vec{s})$-basis,
the sum of the $\vec{s}$-column degrees of $\mathbf{P}_{2}$ satisfies
$\sum\deg_{\vec{s}}\mathbf{P}_{2}\le3sn+\xi=4\xi$ by \ref{lem:boundOfSumOfShiftedDegreesOfOrderBasis}.
Hence \prettyref{lem:multiplyUnbalancedMatrices} applies.
\end{pf}
It remains to show that the residual $\mathbf{F}\mathbf{P}_{2}$ can
be used to compute the remaining nullspace basis elements in \prettyref{lem:continueComputingNullspaceBasisByColumns},
which is based on \prettyref{lem:continueComputingOrderBasis}, a
special case of Theorem 5.1 in \textbf{\citep{BL1997} }also used
in \citep{za2009}. We include a proof here for completeness.
\begin{lem}
\label{lem:continueComputingOrderBasis}For an input matrix $\mathbf{F}\in\mathbb{K}\left[x\right]^{m\times n}$,
an order vector $\sigma$, and a shift vector $\vec{s}$, if $\mathbf{P}$
is a $\left(\mathbf{F},\sigma,\vec{s}\right)$-basis with $\vec{s}$-column
degrees $\vec{s}'$, and $\mathbf{P}'$ is a $(\mathbf{F}\mathbf{P},\tau,\vec{s}')$
-basis with $\vec{s}'$-column degrees $\vec{s}"$ and its order $\tau\ge\sigma$,
then $\mathbf{P}\mathbf{P}'$ is a $(\mathbf{F},\tau,\vec{s})$-basis
with $\vec{s}$-column degrees $\vec{s}"$.\end{lem}
\begin{pf}
It is clear that $\mathbf{P}\mathbf{P}'$ has order $(\mathbf{F},\tau)$.
We now show that $\mathbf{P}\mathbf{P}'$ is $\vec{s}$-column reduced
and has $\vec{s}$-column degrees $\vec{s}"$, or equivalently, $x^{\vec{s}}\mathbf{P}\mathbf{P}'$
is column reduced and has column degrees $\vec{s}"$. Notice that
$x^{\vec{s}}\mathbf{P}$ has column degrees $\vec{s}'$ and a full
rank leading column coefficient matrix $P$. Hence $x^{\vec{s}}\mathbf{P}x^{-\vec{s}'}$
has column degrees $\left[0,\dots0\right]$. (If one is concerned
about the entries not being polynomials, one can simply multiply it
by $x^{\xi}$ to shift the degrees up.) Similarly, $x^{\vec{s}'}\mathbf{P}'x^{-\vec{s}"}$
has column degrees $[0,\dots,0]$ and a full rank leading column coefficient
matrix $P'$. Putting them together, $x^{\vec{s}}\mathbf{P}x^{-\vec{s}'}x^{\vec{s}'}\mathbf{P}'x^{-\vec{s}"}=x^{\vec{s}}\mathbf{P}\mathbf{P}'x^{-\vec{s}"}$
has column degrees $[0,\dots,0]$ and a full rank leading column coefficient
matrix $PP'$. It follows that $x^{\vec{s}}\mathbf{P}\mathbf{P}'$
has column degrees $\vec{s}"$, or equivalently, the $\vec{s}$-column
degrees of $\mathbf{PP}'$ is $\vec{s}"$.

It remains to show that any $\mathbf{t}\in\left\langle \left(\mathbf{F},\tau\right)\right\rangle $
is generated by the columns of $\mathbf{PP}'$. Since $\mathbf{t}\in\left\langle \left(\mathbf{F},\sigma\right)\right\rangle $,
it is generated by the $\left(\mathbf{F},\sigma\right)$-basis $\mathbf{P}$,
that is, $\mathbf{t}=\mathbf{P}\mathbf{a}$ for $\mathbf{a}=\mathbf{P}^{-1}\mathbf{t}\in\mathbb{K}\left[x\right]^{n}$.
Also, $\mathbf{t}\in\left\langle \left(\mathbf{F},\tau\right)\right\rangle $
implies that $\mathbf{a}\in\left\langle \left(\mathbf{FP},\tau\right)\right\rangle $
since $\mathbf{F}\mathbf{P}\mathbf{a}=\mathbf{F}\mathbf{t}\equiv0\mod x^{\tau}$.
It follows that $\mathbf{a}=\mathbf{P}'\mathbf{b}$ for $\mathbf{b}=\mathbf{P}'^{-1}\mathbf{a}\in\mathbb{K}\left[x\right]^{n}$.
Therefore, $\mathbf{a}=\mathbf{P}^{-1}\mathbf{t}=\mathbf{P}'\mathbf{b}$,
which gives $\mathbf{t}=\mathbf{P}\mathbf{P}'\mathbf{b}$.\end{pf}
\begin{lem}
\label{lem:continueComputingNullspaceBasisByColumns}Let $\mathbf{P}=\left[\mathbf{P}_{1},\mathbf{P}_{2}\right]$
be a $\left(\mathbf{F},3s,\vec{s}\right)$-basis with $\vec{s}$-column
degrees $\vec{b}$ and $\mathbf{P}_{1}$ consists of the nullspace
basis elements computed. Let $\vec{b}'$ be the $\vec{s}$-column
degrees of$\mathbf{P}_{2}$. Let $\mathbf{Q}'$ be a $\left(\mathbf{F}\mathbf{P}_{2},\sigma,\vec{b}'\right)$-basis,
where $\sigma$ is at least $3s$ and is large enough for $\mathbf{Q}'=\left[\mathbf{Q},\mathbf{Q}_{2}\right]$
to contain a complete nullspace basis $\mathbf{Q}$ of $\mathbf{F}\mathbf{P}_{2}$,
then $\left[\mathbf{P}_{1},\mathbf{P}_{2}\mathbf{Q}\right]$ is a
$\vec{s}$-minimal nullspace basis of $\mathbf{F}$.\end{lem}
\begin{pf}
Note that 
\[
\mathbf{P}'=\begin{bmatrix}\mathbf{I}\\
 & \mathbf{Q}'
\end{bmatrix}
\]
 is an $\left(\mathbf{F}\mathbf{P},\sigma,\vec{b}\right)$-basis since
$\mathbf{F}\mathbf{P}=\left[\mathbf{0},\mathbf{F}\mathbf{P}_{2}\right]$
and $\mathbf{Q}'$ is an $\left(\mathbf{F}\mathbf{P}_{2},\sigma,\vec{b}'\right)$-basis.
It follows that $\mathbf{P}\mathbf{P}'=\left[\mathbf{P}_{1},\mathbf{P}_{2}\mathbf{Q}'\right]=\left[\mathbf{P}_{1},\mathbf{P}_{2}\mathbf{Q},\mathbf{P}_{2}\mathbf{Q}_{2}\right]$
is a $\left(\mathbf{F},\sigma,\vec{s}\right)$-basis. Hence $\left[\mathbf{P}_{1},\mathbf{P}_{2}\mathbf{Q}\right]$
is a complete $\vec{s}$-minimal nullspace basis of $\mathbf{F}$.
\end{pf}
\prettyref{lem:continueComputingNullspaceBasisByColumns} shows that
the remaining $\vec{s}$-minimal nullspace basis elements $\mathbf{P}_{2}\mathbf{Q}$
can be correctly computed from the residual $\mathbf{F}\mathbf{P}_{2}$.
We still need to show it can be done efficiently. Before discussing
the computation of a $\vec{b}'$-minimal nullspace basis $\mathbf{Q}$
of $\mathbf{F}\mathbf{P}_{2}$, let us first show that the multiplication
$\mathbf{P}_{2}\mathbf{Q}$ can be done efficiently, which is again
a corollary of \prettyref{lem:multiplyUnbalancedMatrices}.
\begin{lem}
\label{cor:multiplyingP2Q}The multiplication of $\mathbf{P}_{2}$
and $\mathbf{Q}$ can be done with a cost of $O^{\sim}\left(nm^{\omega-2}\xi\right)$.\end{lem}
\begin{pf}
Note that the dimension of $\mathbf{P}_{2}$ is $O(m)\times n$ from
\prettyref{lem:dimensionOfPartialNullspaceBasisBasedOnOrder}and column
degrees of $\mathbf{P}_{2}$ are bounded by the $\vec{s}$-column
degrees $\vec{b}'$ of $\mathbf{P}_{2}$ since $\vec{s}$ is non-negative.
By \prettyref{lem:boundOnDegreesOfFA} the column degrees of $\mathbf{F}\mathbf{P}_{2}$
are bounded by the $\vec{s}$-column degrees $\vec{b}'$ of $\mathbf{P}_{2}$
. By \prettyref{lem:boundOfSumOfShiftedDegreesOfNullspaceBasis},
the sum of the $\vec{b}'$-column degrees of $\mathbf{Q}$ is bounded
by the sum of the column degrees of $\mathbf{F}\mathbf{P}_{2}$, which
is bounded by $\sum\vec{b}'\le4\xi$ from \prettyref{cor:multiplyingFP2}.
Now if we separate $\mathbf{F}$ to $n/m$ blocks rows each with no
more than $m$ rows, \prettyref{lem:multiplyUnbalancedMatrices} can
be used to multiply each block row with $\mathbf{P}_{2}$, which costs
$O^{\sim}(m^{\omega-1}\xi)$. Hence doing this for all $n/m$ block
rows costs $O^{\sim}\left(nm^{\omega-2}\xi\right)$.
\end{pf}

\subsection{Reducing the degrees}

Our next task is computing a $\vec{b}'$-minimal nullspace basis of
the residual $\mathbf{F}\mathbf{P}_{2}$. It is useful to note that
the lower degree terms of $\mathbf{F}\mathbf{P}_{2}$ are zero since
it has order $\sigma$. Hence we can use $\mathbf{G}=\mathbf{F}\mathbf{P}_{2}/x^{3s}$
instead to compute the remaining basis elements. But we still need
verify that the size of $\mathbf{G}$ is small enough to allow efficient
computation of the remaining basis elements. \prettyref{lem:boundOnShiftedDegrees}
shows that the $\left(\vec{b}-\sigma\right)$-column degrees of an
$(\mathbf{F},\sigma,\vec{s})$-basis, which bound the column degrees
of $\mathbf{F}\mathbf{P}/x^{\sigma}$ by \prettyref{lem:boundOnDegreesOfFA},
are bounded by the corresponding entries of $\vec{s}$.
\begin{lem}
\label{lem:boundOnShiftedDegrees}Let $\vec{s}$, $\mathbf{F}$ be
as before. If an $(\mathbf{F},\sigma,\vec{s})$-basis has columns
arranged in increasing $\vec{s}$-column degrees with $\vec{s}$-column
degrees $\vec{b}$, then the entries of $\vec{b}-\sigma=\left[b_{1}-\sigma,\dots,b_{n}-\sigma\right]$
are bounded component-wise by $\vec{s}$.\end{lem}
\begin{pf}
An $(\mathbf{F},0,\vec{s})$-basis has $\vec{s}$-column degrees $\vec{s}$.
Then for each order increase, any column of the basis has its $\vec{s}$-column
degree increases by at most one, hence at order $\sigma$, the $\vec{s}$-column
degree increase for each column is at most $\sigma$.
\end{pf}
From \prettyref{lem:boundOnDegreesOfFA}, the column degrees of $\mathbf{F}\mathbf{P}/x^{\sigma}$
are bounded by $\vec{b}-\sigma$, where $\vec{b}$ is the $\vec{s}$-column
degrees of $\mathbf{P}$. Hence the column degrees of $\mathbf{F}\mathbf{P}_{2}/x^{\sigma}$
are bounded by the entries of a corresponding subset $\vec{t}$ of
$\vec{b}-\sigma$, which is in turn bounded by the entries of $\vec{s}$. 

At this point, using \prettyref{lem:continueComputingNullspaceBasisByColumns}
and \prettyref{lem:boundOnShiftedDegrees}, the problem is reduced
to computing a $\vec{t}$-minimal nullspace basis of $\mathbf{G}=\mathbf{F}\mathbf{P}_{2}/x^{3s}$,
which still has row dimension $m$. But its column dimension is now
bounded by $3m/2$. Also notice that as in the original problem, the
column degrees of the new input matrix $\mathbf{G}$ is bounded by
the corresponding entries of the new shift $\vec{t}$. In addition,
bounding the new shift $\vec{t}$ with the old shift $\vec{s}$ ensures
that the new problem is not more difficult than the original problem. 

\begin{comment}
Note the difference between \prettyref{lem:boundOfSumOfShiftedDegreesOfNullspaceBasis}
and \prettyref{lem:boundOnShiftedDegrees}. In \prettyref{lem:boundOfSumOfShiftedDegreesOfNullspaceBasis},
we consider only the shifted degree of a nullspace basis, where as
in \prettyref{lem:boundOnShiftedDegrees} applies to a order basis
but the order is subtracted from the degrees.
\end{comment}



\subsection{\label{sub:continueComputingNullspaceBasisByRows}Reducing the row
dimension }

We now turn to the new problem of computing a $\vec{t}$-minimal nullspace
basis of $\mathbf{G}$. Let 
\[
\left[\begin{array}{c}
\mathbf{G}_{1}\\
\mathbf{G}_{2}
\end{array}\right]=\mathbf{G}
\]
 with $\mathbf{G}_{1}$ having $\left\lfloor m/2\right\rfloor $ rows
and $\mathbf{G}_{2}$ having $\left\lceil m/2\right\rceil $ rows.
If we compute a $\vec{t}$-minimal nullspace basis $\mathbf{N}_{1}$
of $\mathbf{G}_{1}$, where $\mathbf{N}_{1}$ has $\vec{t}$-column
degrees $\vec{u}$, then compute a $\vec{u}$-minimal nullspace basis
$\mathbf{N}_{2}$ of $\mathbf{G}_{2}\mathbf{N}_{1}$, then \prettyref{lem:continueComputingNullspaceBasisByRows}
shows that $\mathbf{N}_{1}\mathbf{N}_{2}$ is a $\vec{t}$-minimal
nullspace basis of $\mathbf{G}$. 
\begin{lem}
\label{lem:continueComputingNullspaceBasisByRows}For an input matrix
$\mathbf{G}=\left[\mathbf{G}_{1}^{T},\mathbf{G}_{2}^{T}\right]^{T}\in\mathbb{K}\left[x\right]^{m\times n}$
and a shift vector $\vec{t}$, if $\mathbf{N}_{1}$ is a $\vec{t}$-minimal
nullspace basis of $\mathbf{G}_{1}$ with $\vec{t}$-column degrees
$\vec{u}$, and $\mathbf{N}_{2}$ is a $\vec{u}$-minimal nullspace
basis of $\mathbf{G}_{2}\mathbf{N}_{1}$ with $\vec{u}$-column degrees
$\vec{v}$, then $\mathbf{N}_{1}\mathbf{N}_{2}$ is a $\vec{t}$-minimal
nullspace basis of $\mathbf{G}$ with $\vec{t}$-column degrees $\vec{v}$.\end{lem}
\begin{pf}
The proof is very similar to the proof of \prettyref{lem:continueComputingOrderBasis}.
It is clear that $\mathbf{G}\mathbf{N}_{1}\mathbf{N}_{2}=0$ hence
$\mathbf{N}_{1}\mathbf{N}_{2}$ is a nullspace basis of $\mathbf{G}$.
We now show that $\mathbf{N}_{1}\mathbf{N}_{2}$ is $\vec{t}$-column
reduced and has $\vec{t}$-column degrees $\vec{v}$, or equivalently,
$x^{\vec{t}}\mathbf{N}_{1}\mathbf{N}_{2}$ is column reduced. Notice
that $x^{\vec{t}}\mathbf{N}_{1}$ has column degrees $\vec{u}$ and
a full rank leading column coefficient matrix $N_{1}$. Hence $x^{\vec{t}}\mathbf{N}_{1}x^{-\vec{u}}$
has column degrees $\left[0,\dots,0\right]$. (If one is concerned
about the entries not being polynomials, one can simply multiply it
by $x^{\xi}$ to shift the degrees up.) Similarly, $x^{\vec{u}}\mathbf{N}_{2}x^{\vec{v}}$
has column degrees $\left[0,\dots,0\right]$ and a full rank leading
column coefficient matrix $N_{2}$. Putting them together, $x^{\vec{t}}\mathbf{N}_{1}x^{-\vec{u}}x^{\vec{u}}\mathbf{N}_{2}x^{-\vec{v}}=x^{\vec{t}}\mathbf{N}_{1}\mathbf{N}_{2}x^{-\vec{v}}$
has column degrees $[0,\dots,0]$ and a full rank leading column coefficient
matrix $N_{1}N_{2}$. It follows that $x^{\vec{t}}\mathbf{N}_{1}\mathbf{N}_{2}$
has column degrees $\vec{v}$, or equivalently, the $\vec{t}$-column
degrees of $\mathbf{N}_{1}\mathbf{N}_{2}$ is $\vec{v}$.

It remains to show that any $\mathbf{n}$ satisfying $\mathbf{G}\mathbf{n}=0$
must be a linear combination of the columns of $\mathbf{N}_{1}\mathbf{N}_{2}$.
First notice that $\mathbf{n}=\mathbf{N}_{1}\mathbf{a}$ for some
polynomial vector $\mathbf{a}$ since $\mathbf{N}_{1}$ is a nullspace
basis of $\mathbf{G}$. Also, $\mathbf{G}\mathbf{n}=0$ implies that
$\mathbf{G}_{2}\mathbf{N}_{1}\mathbf{a}=0$, hence $\mathbf{a}=\mathbf{N}_{2}\mathbf{b}$
for some vector $\mathbf{b}$ as $\mathbf{N}_{2}$ is a nullspace
basis of $\mathbf{G}_{2}\mathbf{N}_{1}$. We now have $\mathbf{n}=\mathbf{N}_{1}\mathbf{N}_{2}\mathbf{b}$
as required.
\end{pf}
While \prettyref{lem:continueComputingNullspaceBasisByColumns} allows
us to compute nullspace basis by columns and then reduce the column
dimensions, \prettyref{lem:continueComputingNullspaceBasisByRows}
shows that that the nullspace basis can also be computed by rows and
then reduce the row dimensions. Again, we need to check these computations
can be done efficiently. \prettyref{lem:sizeOfG2N1} shows that the
size of $\mathbf{G}_{2}\mathbf{N}_{1}$ is within a required bound.
Then \prettyref{lem:mutiplyingG2N1} and \prettyref{lem:multiplyingN1N2}
show that the multiplication $\mathbf{G}_{2}\mathbf{N}_{1}$ and the
multiplication $\mathbf{N}_{1}\mathbf{N}_{2}$ can be done efficiently,
which are again consequences of \prettyref{lem:multiplyUnbalancedMatrices}.
\begin{lem}
\label{lem:sizeOfG2N1}The sum of the column degrees of $\mathbf{G}_{2}\mathbf{N}_{1}$
is bounded by $\xi$.\end{lem}
\begin{pf}
This follows from \prettyref{lem:boundOfSumOfShiftedDegreesOfNullspaceBasis}
and \prettyref{lem:boundOnDegreesOfFA}.\end{pf}
\begin{lem}
\label{lem:mutiplyingG2N1}The multiplication of $\mathbf{G}_{2}$
and $\mathbf{N}_{1}$ can be done with a cost of $O^{\sim}(m^{\omega-1}\xi)$.\end{lem}
\begin{pf}
\prettyref{lem:multiplyUnbalancedMatrices} applies directly here.\end{pf}
\begin{lem}
\label{lem:multiplyingN1N2}The multiplication of $\mathbf{N}_{1}$
and $\mathbf{N}_{2}$ can be done with a cost of $O^{\sim}(m^{\omega-1}\xi)$.\end{lem}
\begin{pf}
\prettyref{lem:multiplyUnbalancedMatrices} applies because the sum
of the column degrees of $\mathbf{N}_{1}$ is bounded by the sum of
the $\vec{t}$-column degrees of $\mathbf{N}_{1}$, which is $\sum\vec{u}\le\xi$,
and by \prettyref{lem:sizeOfG2N1} and \prettyref{lem:boundOfSumOfShiftedDegreesOfNullspaceBasis}
the sum of $\vec{u}$-column degrees of $\mathbf{N}_{2}$ is also
bounded by $\xi$.
\end{pf}

\subsection{Recursive computation}

The computation of $\mathbf{N}_{1}$ and $\mathbf{N}_{2}$ is identical
to the original problem, only the dimension has decreased. For computing
$\mathbf{N}_{1}$, the dimension of the input matrix $\mathbf{G}_{1}$
is bounded by $\left\lfloor m/2\right\rfloor \times\left(3m/2\right)$.
For computing $\mathbf{N}_{2}$ , the dimension of in put matrix $\mathbf{G}_{2}\mathbf{N}_{1}$
is bounded by $\left\lceil m/2\right\rceil \times m$. The sum of
column degrees of $\mathbf{G}_{1}$ and that of $\mathbf{G}_{2}\mathbf{N}_{1}$
are both bounded by $\xi$ still. Hence, the same computation process
can be repeated on these two smaller problems. This gives as a recursive
algorithm, shown in \prettyref{alg:minimalNullspaceBasis}. \begin{algorithm}[t]
\caption{$\mnb(\mathbf{F},\vec{s})$ Compute Minimal Nullspace Basis}


\label{alg:minimalNullspaceBasis} 

\begin{algorithmic}
[1]\REQUIRE{$\mathbf{F}\in\mathbb{K}\left[x\right]^{m\times n}$, $\vec{s}=[s_{1},\dots,s_{n}]\in\mathbb{Z}^{n}$
with entries arranged in non-decreasing order and bound the corresponding
column degrees of $\mathbf{F}$.}

\ENSURE{A $\vec{s}$-minimal nullspace basis of $\mathbf{F}$ and its $\vec{s}$-column
degrees $\vec{s}'$.}

\STATE{$\xi:=\sum_{i=1}^{n}s_{n}$; $\rho:=\sum_{i=n-m+1}^{n}s_{n};$ $s:=\rho/m$; }

\label{line:orderBasis}\STATE{$\left[\mathbf{P},\vec{b}\right]:=\mab\left(\mathbf{F},3s,\vec{s}\right)$,
a $\left(\mathbf{F},3s,\vec{s}\right)$-basis with the columns of
$\mathbf{P}$ and the entries of is $\vec{s}$-column degrees $\vec{b}$
arranged so that the entries of $\vec{b}$ are in non-decreasing order;}

\STATE{$\left[\mathbf{P}_{1},\mathbf{P}_{2}\right]:=\mathbf{P}$ where $\mathbf{P}_{1}$
consists of all columns $\mathbf{p}$ of $\mathbf{P}$ satisfying
$\mathbf{F}\mathbf{p}=0$;}

\STATE{ $\vec{t}=\deg_{\vec{s}}\mathbf{P}_{2}-\left[3s,3s,\dots,3s\right];$}

\label{line:multiplyFP2}\STATE{$\mathbf{G}:=\mathbf{F}\mathbf{P}_{2}/x^{3s}$;}

\STATE{$\left[\mathbf{G}_{1}^{T},\mathbf{G}_{2}^{T}\right]^{T}:=\mathbf{G}$,
with $\mathbf{G}_{1}$ having $\left\lfloor m/2\right\rfloor $ rows
and $\mathbf{G}_{2}$ having $\left\lceil m/2\right\rceil $ rows;}

\STATE{$\left[\mathbf{N}_{1},\vec{u}\right]:=\mnb\left(\mathbf{G}_{1},\vec{t}\right);$ }

\label{line:multiplyG2N1}\STATE{$\left[\mathbf{N}_{2},\vec{v}\right]:=\mnb\left(\mathbf{G}_{2}\mathbf{N}_{1},\vec{u}\right);$}

\label{line:multiplyN1N2}\STATE{$\mathbf{Q}:=\mathbf{N}_{1}\mathbf{N}_{2}$;}

\label{line:multiplyP2Q}\RETURN $\left[\mathbf{P}_{1},\mathbf{P}_{2}\mathbf{Q}\right],\left[\deg_{\vec{s}}\mathbf{P}_{1},\vec{v}\right]$
\end{algorithmic}
\end{algorithm}




\subsection{Overall cost}

In this subsection, we analyze the cost of the algorithm. First we
consider the case where the column dimension $n$ is not much bigger
than the row dimension $m$. 
\begin{lem}
\label{lem:costLowColDimension}If $n\in O\left(m\right)$, then \prettyref{alg:minimalNullspaceBasis}
uses $O^{\sim}\left(m^{\omega-1}\xi\right)=O^{\sim}\left(m^{\omega-1}\rho\right)$
field operations to compute a $\vec{s}$-minimal nullspace basis of
$\mathbf{F}$.\end{lem}
\begin{pf}
We may assume $m$ is a power of 2, which can be achieved by appending
zero rows to $\mathbf{F}$. Note that $\rho\in\Theta\left(\xi\right)$
when $n\in O\left(m\right)$. Then \prettyref{line:orderBasis} costs
$O^{\sim}\left(n^{\omega}s\right)=O^{\sim}\left(m^{\omega-1}\rho\right)$;
\prettyref{line:multiplyFP2} and \prettyref{cor:multiplyingP2Q}cost
$O^{\sim}\left(nm^{\omega-2}\xi\right)=O^{\sim}\left(m^{\omega-1}\xi\right)$.
The remaining operations including multiplications on \prettyref{line:multiplyG2N1},
\prettyref{line:multiplyN1N2}, and \prettyref{line:multiplyP2Q}
cost $O^{\sim}\left(m^{\omega-1}\xi\right)$. Let $g(m,\xi)$ be the
computational cost of the original problem, then we have the recurrence
relation 
\[
g(m,\xi)\in O^{\sim}(m^{\omega-1}\xi)+g(m/2,\xi)+g(m/2,\xi),
\]
 which gives $g(m,\xi)\in O^{\sim}(m^{\omega-1}\xi)$. Note that $O^{\sim}\left(m^{\omega-1}\xi\right)=O^{\sim}\left(m^{\omega-1}\rho\right)$
since $\rho\in\Theta\left(\xi\right)$ when $n\in O\left(m\right)$.
\end{pf}
We now consider the general case where the column dimension $n$ can
be much bigger than the row dimension $m$.
\begin{lem}
\label{lem:costGeneral}\prettyref{alg:minimalNullspaceBasis} cost
$O^{\sim}\left(n^{\omega}s\right)$ field operations in general.\end{lem}
\begin{pf}
The order basis computation on \prettyref{line:orderBasis} cost $O^{\sim}\left(n^{\omega}s\right)$
in general, which dominates the cost of other operations. The problem
is then reduced to have column dimension $O\left(m\right)$, which
is handled by \prettyref{lem:costLowColDimension} with a cost of
$O^{\sim}\left(m^{\omega-1}\xi\right)\in O^{\sim}\left(n^{\omega}s\right)$.
\end{pf}
We now consider the important special case where the shift $\vec{s}=\left[s,\dots,s\right]$
is uniform. \prettyref{alg:minimalNullspaceBasis} has a lower cost
in this case.

First notice that \prettyref{line:orderBasis} cost $O^{\sim}\left(n^{\omega}\left\lceil ms/n\right\rceil \right)$
using the algorithm from \citet{za2009}. In addition, the multiplication
of $\mathbf{F}$ and $\mathbf{P}_{2}$ at \prettyref{line:multiplyFP2}
and the multiplication of $\mathbf{P}_{2}$ and $\mathbf{Q}$ at \prettyref{line:multiplyP2Q}
cost $O^{\sim}\left(nm^{\omega-1}s\right)$ as shown in \prettyref{lem:multiplyFP2WithUniformShift}
and \prettyref{lem:multiplyP2QWithUniformShift}.
\begin{lem}
\label{lem:multiplyFP2WithUniformShift}If the degree of $\mathbf{F}$
is bounded by $s$, then the degree of $\mathbf{P}_{2}$ is bounded
by $3s$, and the multiplication of $\mathbf{F}$ and $\mathbf{P}_{2}$
at \prettyref{line:multiplyFP2} costs \textup{$O^{\sim}\left(nm^{\omega-1}s\right)$.}\end{lem}
\begin{pf}
Since $\mathbf{P}_{2}$ is a part of a $\left(\mathbf{F},3s,\vec{s}\right)$-basis,
its degree is bounded by $3s$. It has dimension $n\times O\left(m\right)$
from \prettyref{lem:dimensionOfPartialNullspaceBasisBasedOnOrder}.
Multiplying $\mathbf{F}$ and $\mathbf{P}_{2}$ therefore costs $(n/m)O^{\sim}\left(m^{\omega}s\right)=O^{\sim}\left(nm^{\omega-1}s\right)$.\end{pf}
\begin{lem}
\label{lem:multiplyP2QWithUniformShift}If $\mathbf{F}$ has degree
$s$, then the multiplication of $\mathbf{P}_{2}$ and $\mathbf{Q}$
at \prettyref{line:multiplyP2Q} costs \textup{$O^{\sim}\left(nm^{\omega-1}s\right)$.}\end{lem}
\begin{pf}
First note that the dimension of $\mathbf{Q}$ is $O\left(m\right)\times O\left(m\right)$
since it is a $\vec{t}$-minimal nullspace basis of $\mathbf{G}=\mathbf{F}\mathbf{P}_{2}/x^{3s}$,
which has dimension $m\times O\left(m\right)$. In addition, by \prettyref{lem:boundOfSumOfShiftedDegreesOfNullspaceBasis},
the sum of the $\vec{t}$-column degrees of $\mathbf{Q}$ is bounded
by $\sum\vec{t}$, which is bounded by $O\left(ms\right)$ since $\vec{t}$
has $O\left(m\right)$ entries all bounded by $s$.

Now \prettyref{lem:multiplyUnbalancedMatrices} and its proof still
work. The current situation is even simpler as we do not need to subdivide
the columns of $\mathbf{P}_{2}$, which has degree bounded by $3s$
and dimension $n\times O\left(m\right)$. We just need to separate
the columns of $\mathbf{Q}$ to $O\left(\log m\right)$ groups with
degree ranges $\left[0,\dots,2s\right],(2s,4s],(4s,8s]\dots$ , and
multiply $\mathbf{P}_{2}$ with each group in the same way as in \prettyref{lem:multiplyUnbalancedMatrices},
with each of these $O\left(\log m\right)$ multiplications costs $(n/m)O^{\sim}\left(m^{\omega}s\right)=O^{\sim}\left(nm^{\omega-1}s\right)$.\end{pf}
\begin{lem}
\label{lem:costOfMinimalNullspaceBasisWithUniformShift}If $\vec{s}=\left[s,\dots,s\right]$
is uniform, then \prettyref{alg:minimalNullspaceBasis} costs $O^{\sim}\left(n^{\omega}\left\lceil ms/n\right\rceil \right)$. \end{lem}
\begin{pf}
After the initial order basis computation, which costs $O^{\sim}\left(n^{\omega}\left\lceil ms/n\right\rceil \right),$
and the multiplication of $\mathbf{F}$ and $\mathbf{P}_{2}$, which
costs $O^{\sim}\left(nm^{\omega-1}s\right)$ from \prettyref{lem:multiplyFP2WithUniformShift},
the column dimension is reduced to $O\left(m\right)$, allowing \prettyref{lem:costLowColDimension}
to apply for computing a . Hence the remaining work costs $O^{\sim}\left(m^{\omega}s\right)$.
The overall cost is therefore dominated by the cost $O^{\sim}\left(n^{\omega}\left\lceil ms/n\right\rceil \right)$
of the initial order basis computation.\end{pf}
\begin{thm}
\label{cor:costOfMinimalNullspaceBasis}If the input matrix $\mathbf{F}$
has degree $d$, then a minimal nullspace basis of $\mathbf{F}$ can
be computed with a cost of $O^{\sim}\left(n^{\omega}\left\lceil md/n\right\rceil \right)$. \end{thm}
\begin{pf}
We can just set the shift $\vec{s}$ to $\left[d,\dots,d\right]$
and apply \prettyref{lem:costOfMinimalNullspaceBasisWithUniformShift}.\end{pf}
\begin{rem}
One may wonder whether it is possible slightly improve the algorithm
to remove the ceiling function in the cost $O^{\sim}\left(n^{\omega}\left\lceil md/n\right\rceil \right)$,
to get a cost of $O^{\sim}\left(n^{\omega-1}md\right)$. Unfortunately,
it does not seem possible. Notice that the ceiling function is only
used to take care of the case $md<n$. In this case, the sum of the
size of the input and the size of the output is at least $n^{2}$,
which may already be greater than $n^{\omega-1}md$.\end{rem}




\section{Future Research\label{sec:Future-Research}}

The algorithms in this paper give fast procedures for computing minimal
nullspace bases. The problem comes naturally after this is the problem
of computing reduced matrix GCDs, or equivalently, the problem of
column reduction, which then leads to the computation of normal forms.
It would be interesting to try applying the ideas from this paper
to these problems.


\bibliographystyle{plainnat}
\bibliography{paper}

\end{document}
